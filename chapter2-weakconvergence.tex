\chapter{Weak Convergence}
% I'd ideally prefer not to refer to pages by hard reference, but I don't have access to the latex and would rather not get it from Varadhan right now.
% Start each page that you need to refer to in a new section
\section{Page 19}
We'll use the shorthand CF to represent characteristic function because of its frequent use in this chapter. The symbol $\phi(t)$ is most often used to represent the CF. The first exercise is straightforward calculation.
\begin{exer}
  \begin{enumerate}
    \item The CF of degenerate distribution at $a$ is
      \[ \int e^{i t x} d \delta_a(x) = e^{ita} \]
    \item The CF of the binomial distribution is
      \[ \sum_k \binom{n}{k} p^k (1-p)^{n-k} e^i t k = (p e^{it} + 1-p)^n \]
  \end{enumerate}
\end{exer}

Theorem $2.1$ and it's proof are clearly explained.

\begin{exer}
  Show that if $\int |x| d\alpha(x) < \infty$, then $\phi(t)$ is continuously differentiable and $\phi'(0) = i \int x d\alpha$.

  We first have to show that $\phi(t)$ is differentiable and so we consider the quotient,
  \begin{equation}
    \frac{\phi(t+\Delta t) - \phi(t)}{\Delta t} = \int e^{i t x} \left( \frac{e^{i \delta t x} - 1}{\delta t} \right) d\alpha 
    \label{eq:difference-quotient-CF}
  \end{equation}
  Taylor's theorem in complex analysis says that since $e^{iz}$ is an entire function (analytic in the entire plane)
  \begin{equation}
    e^{iz} = 1 + i z + O(z^2) 
  \end{equation}
  in a fixed neighbourhood of the origin. So, we can dominate the integrand on the RHS of~\Eqref{eq:difference-quotient-CF} as
  \[|e^{i t x} \left( \frac{e^{i \delta t x} - 1}{\delta t} \right)| \leq 
\end{exer}
